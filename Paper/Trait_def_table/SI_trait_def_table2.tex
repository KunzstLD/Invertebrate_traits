\documentclass[../Draft_harmonization_paper.tex]{subfiles}



\begin{document}

\begin{landscape}
    \begin{longtable}{m{1.7cm}|m{3cm}|m{3cm}|m{3cm}|m{3cm}|m{3cm}|m{3cm}}
        \caption{Comparison of trait definitions between invertebrate trait databases. Only traits that are differently described across databases are listed. The definition is quoted if it enables differences to be identified, otherwise the differences are described. The hyphen indicates a missing trait. Reproduction was captured in multiple grouping features per database. Hence, differences for reproduction have been described in the paper. Body form traits are not different between databases, except that the North America (Vieira) database contains the trait Bluff (blocky) which does not appear in the other databases.}
        \label{stab:trait_definitions}
        \endfirsthead
        \toprule[.1em]
        Trait & Freshwaterecology & Tachet & \specialcell{North America \\ (Twardochleb)} & 
        North America (Vieira) & Australia & New Zealand \\
        \toprule[.1em]
        Feeding shredder & 
        "Feed from fallen leaves, plant tissues, CPOM" & 
        "Eat coarse detritus, plants or \textit{animal material}" & 
        \begin{itemize}
            \item "Shred decomposing vascular plant tissue"
            \item Trait herbivore includes among others insect that shred \textit{living aquatic plants} 
        \end{itemize} & 
        Shredder & 
        \begin{itemize}
            \item Detrivore \textsuperscript{\textit{a}}
            \item Trait herbivore includes among others the trait shredder
        \end{itemize} & 
        Shredders
        \\ 
        \midrule
        Feeding predator & 
        "Eating from prey" & 
        \begin{itemize}
            \item Carvers, engulfers \& swallowers
            \item Piercers (plants \& animals) are an additional trait
        \end{itemize} & % Notes: Tachet -> Piercer (plants & animals)
        Engulfers ("ingest prey whole or in parts") \& 
        piercers ("prey tissues and suck fluids") & 
        Predator &
        Piercer \& engulfer &
        Predator
        \\ 
        \midrule
        Feeding filter-feeder & 
        Distinguishes between active and passive &
        No distinction between active and passive &
        No distinction between active and passive &
        No distinction between active and passive &
        No distinction between active and passive &
        No distinction between active and passive
        \\
        \toprule[.1em]
        Semivoltine & 
        "One generation in two years" & 
        "Life cycle lasts \textit{at least} two years" & 
        "$< 1$ generation per year" & 
        "$< 1$ generation per year" & 
        "$< 1$ generation per year" & 
        "$< 1$ reproductive cycle per year"
        \\
        \midrule
        Multivoltine & 
        "More than \textit{three} generations per year" \textsuperscript{\textit{b}}& 
        "Able to complete \textit{at least} two successive generations per year" &
        "$> 1$ generations per year" &
        "$> 1$ generations per year" & 
        \begin{itemize}
            \item 1-2 generations per year
            \item bi/multivoltine
            \item up to 5 generations per year
            \item up to 10 generations per year
        \end{itemize}
        & 
        "$> 1$ reproductive cycles per year"
        \\
        \toprule[.1em]
        Locomotion swimming & 
        \begin{itemize}
            \item Passive movement like floating or drifting (trait swimming/scating)
            \item Active movement (trait swimming/diving)
        \end{itemize}. &
        \begin{itemize}
            \item Surface swimmers (over and under the water surface)
            \item Full water swimmers (e.g. Baetidae).
        \end{itemize} & 
        "Adapted for "fishlike" swimming" & 
        Swimmer & 
        Distinguishes swimmer and skater & 
        Swimmers (water column)
        \\
        \midrule
        Locomotion burrowing & 
        "Burrowing in \textit{soft} substrates or boring in \textit{hard} substrates" & 
        \begin{itemize}
            \item Burrowing "within the first centimeters of the benthic fine sediment"
            \item Differentiates also the trait interstitial (endobenthic)
        \end{itemize} & 
        "Inhabiting \textit{fine} sediment of streams and lakes" &
        Burrower & 
        "Moving deep into the substrate and thus avoiding flow" &
        Burrowers (infauna)
        \\
        \midrule
        Locomotion sprawling \& walking & 
        "Sprawling or walking actively with legs, pseudopods or on a mucus" &
        - & 
        Sprawling: "inhabiting the surface of floating leaves of vascular hydrophytes or fine sediments" & 
        Sprawler &
        - & 
        - \\
        \midrule
        Locomotion crawling & 
        - &
        "Crawling over the bottom substrate" & 
        Defined as crawling on the surface of floating leaves or fine sediments on the bottom & 
        - & 
        Database contains traits crawler, 
        sprawler, climber and clinger. &
        Crawlers (epibenthic) \\
        \midrule
        Locomotion sessil & 
        Does not distinguish temporarily and permanently attached & 
        Distinguishes temporarily and permanently attached & 
        Does not distinguish temporarily and permanently attached & 
        Does not distinguish temporarily and permanently attached & 
        Distinguishes temporarily and permanently attached & 
        Does not distinguish temporarily and permanently attached \\
        \toprule[.1em]
        Respiration plastron \& spiracle & 
        Plastron and spiracle (aerial) are two separate traits & 
        Definition includes respiration using air stores of aquatic plants & 
        Plastron and spiracle combined into one trait & 
        Distinguishes spiracular gills, plastron, atmospheric breathers and plant breathers &
        Plastron and spiracle (termed aerial) occur as separate and combined traits. Contains also traits: air (plants), atmospheric, and functional spiracles &
        Distinguishes plastron and spiracle (termed aerial) \\
        \toprule[.1em]
        Body size small & 
        - &
        \multirow{3}{*}{\specialcell{Multiple size \\ classifications \textsuperscript{\textit{d}}}} & 
        $<$ 9 mm & 
        $<$ 9 mm & 
        $<$ 9 mm \textsuperscript{\textit{a;c}} &
        \multirow{3}{*}{\specialcell{Multiple size \\ classifications \textsuperscript{\textit{e}}}}
        \\
        \cline{1-2}
        \cline{4-6}
        Body size medium & 
        - &
        &
        9 - 16 mm & 
        9 - 16 mm & 
        9 - 16 mm &
        \\
        \cline{1-2}
        \cline{4-6}
        Body size large & 
        - &
        &
        $>$ 16 mm &
        $>$ 16 mm &
        $>$ 16 mm &
        \\
        \bottomrule
    \end{longtable}
    \begin{minipage}{\linewidth}\small
        \textit{a} Traits from Botwe et al.
        \newline
        \textit{b} Contains also bivoltine (two generations per year), trivoltine (three generations per year) and flexible.
        \newline
        \textit{c} Contains a size trait with numeric size values. Contains also traits classifying size like Tachet and like the North American trait databases. 
        \newline
        \textit{d} Size classifications: \textit{$<=0.25$ cm, $> 0.25-0.5$ cm, $0.5-1$ cm, $1-2$ cm, $2-4$ cm, $4-8$ cm, $> 8$ cm}. No distinction into small, medium and large.
        \newline
        \textit{e} Size classifications: \textit{$> 0.25-0.5$ cm, $0.5-1$ cm, $1-2$ cm, $2-4$ cm, $4-8$ cm}. No distinction into small, medium and large.
    \end{minipage}
\end{landscape}

%     North America (Vieira) & 
%     Body Form & 
%     5 & 
%     categorical & 
%     \begin{itemize}
%         \item Classifies body form using the traits streamlined, dorsoventrally flattened, tubular, round, and bluff (blocky)
%     \end{itemize} & 
%     No precise trait definitions given
%     \\
%     \hline 
%     New Zealand & 
%     Body Form & 
%     4 & 
%     fuzzy codes $[0-3]$ & 
%     \begin{itemize}
%         \item Distinguishes streamlined, flattened, cylindrical, spherical
%     \end{itemize} & 
%     No precise trait definitions given 
%     \\ 
%     \hline
%     Australia & 
%     Body Form & 
%     4 & 
%     fuzzy codes $[0-3]$
%     &
%     \begin{itemize}
%         \item Identical to New Zealand trait database
%     \end{itemize} & 
%     No precise definitions given



% \specialcell{
%         size classifications: \\ $<=0.25$ cm, \\ $> 0.25-0.5$ cm, \\ $0.5-1$ cm , \\ $1-2$ cm, \\ $2-4$ cm, \\ $4-8$ cm, \\ $> 8$ cm}

%%%%%%%%%%%%%%%%%%%%%%%%%%%%%%%%%%%%%%%% OLD VERSION %%%%%%%%%%%%%%%%%%%%%%%%%%%
% \begin{landscape}
%     \begin{longtable}{m{2.5cm} | m{2cm} | m{0.7cm} | m{3cm} | m{7.5cm} | m{3.5cm}}
%     % \centering
%     \caption{Overview of discrepancies between trait definitions of the investigated trait databases. Grouping feature size was not available for the Freshwaterecology database. Grouping feature body form was not available for the Freshwaterecology, Tachet and North American (Twardochleb) databases.}
%     % This is the place for \label
%     \endfirsthead
%     \endhead
%     % \begin{tabular}{ m{2.5cm} | m{2cm} | m{1.5cm} | m{3cm} | m{5cm} | m{5cm}}
%     \hline
%     Database & Grouping feature & Nr. of traits & Coding & Definition differences & Notes \\
%     \hline
%     Freshwater-ecology & 
%     Feeding Mode & 
%     10 &
%     10 point assignment system & 
%     \begin{itemize}
%         \item Shredders: feed from fallen leaves, plant tissue, CPOM
%         \item Predator: eating from prey
%         \item Filter feeders: distinguish between passive and active filterers
%     \end{itemize} & 
%     Contains traits miners and xylophagus, which can be assigned to the shredders
%     \\
%     \hline
%     Tachet & 
%     Feeding Mode & 
%     7 & 
%     fuzzy $[0 - 3]$ &
%     \begin{itemize}
%         \item Shredders: eat coarse detritus, plant or animal material.
%         \item Predators: carvers, engulfers \& swallowers
%         \item Filter-feeders: no distinction between active and
%         passive filterers
%     \end{itemize}
%     &
%     Piercers (plant \& animals) are an additional trait \\
%     %\hline
%     North America (Twardochleb) &
%     Feeding Mode &
%     6 & 
%     categorical & 
%     \begin{itemize}
%         \item Shredder: insects that shred decomposing vascular plant tissue
%         \item Predator: engulfers \& piercers (prey tissues)
%         \item Filter-feeders: no distinction between active and passive
%     \end{itemize} &
%     Trait herbivore includes among others insects that shred \textit{living} aquatic plants. \newline
%     \newline
%     Gatherers denoted as deposit-feeders
%     \\
%     \hline
%     North America (Vieira) & 
%     Feeding Mode & 
%     8 & 
%     categorical & 
%     \begin{itemize}
%         \item Similar trais compared to North America (Twardochleb), but contains traits Scraper/grazer and Piercer herbivore instead of trait herbivore. 
%     \end{itemize} & 
%     Grouping feature defined as feeding guild describing primary mode of food collection \newline
%     \newline 
%     No precise trait definitions given
%     \\
%     \hline
%     New Zealand & 
%     Feeding Mode & 
%     6 & 
%     fuzzy $[0 - 3]$ & 
%     \begin{itemize}
%         \item Filter-feeders: no distinction between active and passive filterers
%     \end{itemize}
%     &
%     Trait parasite is not assessed
%     \newline
%     \newline
%     No precise definitions given \\
%     \hline
%     Australia & 
%     Feeding Mode & 
%     16\textsuperscript{[1]} & 
%     binary (0 or 1); proportional scale $[0 - 1]$; fuzzy $[0 - 3]$ & 
%     \begin{itemize}
%         \item Shredder: detrivore\textsuperscript{[2]}
%         \item Predator: piercer \& engulfer 
%         \item Filter-feeder: no distinction between active and passive filterers
%     \end{itemize} & 
%     No precise definitions given.
%     \newline
%     \newline
%     Trait herbivore includes among others the trait shredder.\\
%     %\hhline{|=|=|=|=|=|=|}
%     Freshwater-ecology & 
%     Voltinism & 
%     6 &
%     single category assignment system & 
%     \begin{itemize}
%         \item Semivoltine: defined as one generation in two years
%         \item Three traits are used to classify taxa with two or more generations per year.
%         \item Multivoltine: taxa are defined as having \textit{more than three} generations per year.
%     \end{itemize} & 
%     Taxa are classified according to different floristic regions (arctic, boreal, ...)
%     \newline
%     \newline
%     Contains as only database a trait for taxa with flexible number of life cycles per year
%     \\
%     \hline
%     Tachet &
%     Voltinism &
%     3 &
%     fuzzy codes $[0-3]$ &
%     \begin{itemize}
%         \item Semivoltine: defined as one generation (life cycle) in \textit{at least} two years
%         \item Multivoltine: defined as \textit{at least} two successive generations per year
%     \end{itemize} &
%     Semantic: multivoltine is termed polyvoltine \\
%     %\hline
%     North America (Twardochleb) &
%     Voltinism &
%     3 &
%     categorical & 
%     \begin{itemize}
%         \item  Semivoltine: defined as $<1$ generations per year
%         \item Multivoltine: defined as $> 1$ generations per year
%     \end{itemize}
%     & 
%     \\
%     \hline 
%     North America (Vieira) & 
%     Voltinism & 
%     3 & 
%     categorical & 
%     \begin{itemize}
%         \item Voltinism classification identical to North America (Twardochleb) database
%     \end{itemize}
%     \\
%     % \hline
%     New Zealand &
%     Voltinism &
%     3 &
%     fuzzy codes $[0-3]$
%     &
%     \begin{itemize}
%         \item semivoltine: defined as $<1$ reproductive cycles per year
%         \item multivoltine: defined as $> 1$ reproductive cycles per year. 
%     \end{itemize}
%     & Multivoltine is termed plurivoltine 
%     \newline
%     \newline
%     The authors note that this grouping feature varies with temperature and latitude
%     \\
%     \hline
%     Australia &
%     Voltinism &
%     7 & 
%     binary (0 or 1); proportional scale $[0 - 1]$; fuzzy $[0 - 3]$ &
%     \begin{itemize}
%         \item Multiple generations per year are captured in several traits (depending on the initial database from which they were taken): 1-2 generations per year, bi/multivoltine, up to 5 generations per year, up to 10 generations per year
%     \end{itemize} &
%     \\
%     \newpage
%     % \hhline{|=|=|=|=|=|=|}
%     Freshwater-ecology & 
%     Locomotion &
%     6 &
%     10 point assignment system &
%     \begin{itemize}
%         \item Swimming: database distinguishes passive movement (floating or drifting) using the trait swimming/scating and active movement using the trait swimming/diving.
%         \item Burrowing/boring: defined for \textit{soft and hard} substrate 
%         \item Sprawling and walking combined into one trait
%     \end{itemize}
%     &
%     Contains a trait \textit{other} for locomotion types like flying or jumping
%     \\
%     \hline
%     Tachet &
%     Locomotion &
%     8 &
%     fuzzy codes $[0-5]$ &
%     \begin{itemize}
%         \item Swimming: distinguishes surface swimmers (over and under the water surface) and full water swimmers (e.g. Baetidae)
%         \item Burrowing: defined as burrowing within the first centimeters of the benthic fine sediment.
%         \item Crawler: defined as crawling over the bottom substrate. Traits sprawlers or walkers do not exist. 
%         \item Sessil: captured in the two traits temporarily and permanently attached. Freshwaterecology, both North American databases and the New Zealand database do not make this distinction.
%     \end{itemize} &
%     Contains a trait \textit{flier} encompassing active and passive fliers. Also contains a trait interstitial (endobenthic)
%     \\
%     %\hline
%     North America 
%     (Twardochleb) & 
%     Locomotion &
%     10 &
%     categorical &
%     \begin{itemize}
%         \item Swimming: defined as adapted for "fishlike" swimming. 
%         \item Skating: separately defined as organisms that skate on the surface where they feed on organisms trapped in the surface film.
%         \item Burrowing: defined as inhabiting \textit{fine} sediment of streams and lakes
%         \item Crawler: defined for crawling on bottom sediment \textit{and} on floating leaves.
%     \end{itemize} &
%     Grouping feature termed habit %and is based on how organisms deal with flow
%     \newline
%     \newline
%     The grouping feature also comprises the traits clinger, sprawler, climber, planktonic and other. \\
%     \hline
%     North America (Vieira 2006) &
%     Locomotion &
%     9 &
%     categorical &
%     \begin{itemize}
%         \item Contains identical traits as the North America (Twardochleb) database, except that it does not contain the trait crawler.
%     \end{itemize} &
%     Grouping feature termed habit. Is based on how organisms deal with flow.
%     \\
%     \hline
%     New Zealand &
%     Locomotion &
%     4 & 
%     fuzzy codes &
%     \begin{itemize}
%         \item Databases uses the four traits swimmer, crawler, burrower and attached.
%     \end{itemize}
%     & 
%     Grouping feature termed attachment to substrate of aquatic stages (excluding eggs)
%     \newline
%     \newline
%     No precise definitions given \\
%     %\hline
%     Australia &
%     Locomotion &
%     9 &
%     binary (0 or 1); fuzzy $[0 - 3]$ & 
%     \begin{itemize}
%         \item Swimming: distinguishes swimmer and skater
%         \item Crawling: database contains traits crawler, sprawler, climber and clinger
%         \item Sessil: distinguishes temporary and permanently attached
%     \end{itemize}
%     & 
%     No precise definitions given
%     \newline
%     \newline
%     Flying and swimming ability are assessed as well, but are considered part of the grouping feature dispersal. \\
%     \hhline{|=|=|=|=|=|=|}
%     Freshwater-ecology & 
%     Respiration & 
%     7 &
%     presence/absence assignment system &
%     \begin{itemize}
%         \item Plastron and spiracle (aerial) are two separate traits
%     \end{itemize}
%     &
%     Contains the trait hydrostatic vesicle.
%     \newline
%     \newline
%     Contains respiration using air stores of aquatic plants (tapping) and excursion/extension to the surface which can be allocated to the taxa with spiracles.
%     \\
%     \hline 
%     Tachet &
%     Respiration &
%     5 &
%     fuzzy codes $[0-3]$ & 
%     \begin{itemize}
%         \item Spiracle trait definition includes respiration using air stores of aquatic plants
%     \end{itemize}
%     & 
%     Contains the trait hydrostatic vesicle.
%     \\
%     \hline
%     North America (Twardochleb) & 
%     Respiration &
%     3 &
%     categorical &
%     \begin{itemize}
%         \item Plastron and spiracle defined as one trait
%     \end{itemize} 
%     & 
%     \\
%     %\hline
%     North America (Vieira 2006) &
%     Respiration &
%     8 &
%     categorical & 
%     \begin{itemize}
%         \item Distinguishes temporary air store and tracheal gills
%         \item Distinguishes spiracular gills, plastron, atmospheric breathers, and plant breathers
%     \end{itemize} &
%     No precise definition of traits given.
%     \newline
%     \newline
%     Contains the trait Hemoglobin, which is not compatible with the respiration traits occurring in the other trait databases.
%     \\
%     \hline
%     New Zealand &
%     Respiration &
%     4 & 
%     fuzzy codes $[0-3]$ & 
%     \begin{itemize}
%         \item Distinguishes plastron and spiracle (here termed aerial)
%     \end{itemize} &
%     No precise definition given \\
%     \hline 
%     Australia & 
%     Respiration & 
%     10 & 
%     binary (0 or 1); proportional scale $[0 - 1]$ ; fuzzy $[0 - 3]$ & 
%     \begin{itemize}
%         \item Plastron and spiracle (here termed aerial) occur as two separate traits and as combined trait
%         \item Contains the traits air (plants), atmospheric, functional spiracles, plastron 
%     \end{itemize}
%     & 
%     No precise definitions given 
%     \newline
%     \newline
%     Contains the traits "plastron and gills" and pneomostome which are not compatible with the respiration traits occurring in the other trait databases.
%     \\
%     \hhline{|=|=|=|=|=|=|}
%     Freshwater-ecology &
%     Reproduction & 
%     9 &
%     presence/absence assignment system &
%     \begin{itemize}
%         \item Form and location of eggs described and mode of reproduction
%         \item Locations: freely in water, fixed, vegetation, riparian zone
%     \end{itemize} &
%     Includes parasitic reproduction 
%     \\
%     %\hline
%     Tachet & 
%     Reproduction &
%     8 &
%     fuzzy codes $[0-3]$ & 
%     \begin{itemize}
%         \item Form and location of eggs described and mode of reproduction
%     \end{itemize}
%     & 
%     Grouping feature combines mode of reproduction and oviposition tactics
%     \\
%     \hline
%     North America (Vieira) & 
%     Reproduction & 
%     10 & 
%     categorical & 
%     \begin{itemize}
%         \item Traits describe the location of oviposition precisely (e.g. algal mats, bank soil, ...)
%         \item Ovoviviparous taxa are mentioned in a comment column
%     \end{itemize} &
%     No precise definitions given
%     \newline
%     \newline 
%     Grouping feature termed oviposition behavior
%     \\
%     \hline 
%     New Zealand & 
%     Reproduction & 
%     4 & 
%     fuzzy codes $[0-3]$ & 
%     \begin{itemize}
%         \item Information captured in two grouping features: oviposition site using four traits and egg/egg mass describing if eggs are free, cemented, or in/on body.
%         \item Oviposition site defined more broadly than in the other databases (water surface, submerged, terrestrial, endophytic)
%     \end{itemize} & 
%     No precise definitions given 
%     \\
%     \hline
%     Australia &
%     Reproduction &
%     13\textsuperscript{[3]} & 
%     categorical 
%     & 
%     \begin{itemize}
%         \item Contains traits that precisely describe oviposition location % some rather comments
%         \item Also contains traits broadly describing oviposition (aquatic eggs, terrestrial eggs, ovoviviparity)
%         \item Contains traits for mode of reproduction
%     \end{itemize} & 
%     No precise definitions given 
%     \\
%     \newpage
%     % \hhline{|=|=|=|=|=|=|}
%     Tachet & 
%     Size & 
%     7 & 
%     fuzzy codes $[0-3]$ & 
%     \begin{itemize}
%         \item Describes the maximal potential size that
%         \textit{the last} aquatic stage of the taxon can reach. All other databases describe the maximal potential size without mentioning the aquatic stage.
%         \item Highest resolved size classification: $<=0.25 cm, > 0.25-0.5 cm, 0.5-1 cm , 1-2 cm, 2-4 cm, 4-8 cm, >8 cm$
%         \item Draws no threshold what a small, medium and large organism is
%     \end{itemize} &
%     \\
%     \hline 
%     North America
%     (Twardochleb) &
%     Size &
%     3 &
%     categorical & 
%     \begin{itemize}
%         \item Defines small ($< 9 mm$), medium ($9-16 mm$) and large ($> 16 mm$). 
%     \end{itemize}
%     & 
%     Size classification uses not the same ranges as the size classifications in the Tachet and New Zealand trait databases.
%     \\
%     \hline
%     North America (Vieira) &
%     Size & 
%     3 &
%     categorical &
%     \begin{itemize}
%         \item Identical to North America (Twardochleb)
%     \end{itemize} 
%     &
%     \\
%     \hline
%     New Zealand & 
%     Size & 
%     5 & 
%     fuzzy codes $[0-3]$ &
%     \begin{itemize}
%         \item Size traits similarly defined to Tachet, but without the smallest and largest size traits
%         \item Draws no threshold what a small, medium and large organism is
%     \end{itemize} & 
%     \\
%     %\hline 
%     Australia &
%     Size & 
%     9
%     &
%     Numeric, fuzzy $[0 - 3]$,
%     binary (0 or 1) & 
%     \begin{itemize}
%         \item Contains a size trait with numeric size values
%         \item Contains traits classifying size like Tachet (except highest and lowest) and like North American trait databases
%     \end{itemize} & 
%     \\
%     \hhline{|=|=|=|=|=|=|}
%     North America (Vieira) & 
%     Body Form & 
%     5 & 
%     categorical & 
%     \begin{itemize}
%         \item Classifies body form using the traits streamlined, dorsoventrally flattened, tubular, round, and bluff (blocky)
%     \end{itemize} & 
%     No precise trait definitions given
%     \\
%     \hline 
%     New Zealand & 
%     Body Form & 
%     4 & 
%     fuzzy codes $[0-3]$ & 
%     \begin{itemize}
%         \item Distinguishes streamlined, flattened, cylindrical, spherical
%     \end{itemize} & 
%     No precise trait definitions given 
%     \\ 
%     \hline
%     Australia & 
%     Body Form & 
%     4 & 
%     fuzzy codes $[0-3]$
%     &
%     \begin{itemize}
%         \item Identical to New Zealand trait database
%     \end{itemize} & 
%     No precise definitions given
%     \\
%     \hline
%     \end{longtable}
%     \footnotemark{\textit{Many traits were just semantically different.}} 
%     \newline
%     \footnotemark{\textit{Traits from Botwe et al.}}
%     \newline
%     \footnotemark{\textit{Few traits not counted due to their ambigious nature (rather comments).}}
% \end{landscape}
% % TODO Fix AUS DB with Reproduction traits

\end{document}