\documentclass[../Draft_harmonization_paper.tex]{subfiles}
% \documentclass{article}
% \usepackage[utf8]{inputenc}
% \usepackage{hyperref}
% \usepackage{float}
% \usepackage[table,xcdraw]{xcolor}
% \usepackage{color, colortbl}
% \usepackage{longtable}
% %\usepackage[sort&compress,square,comma,authoryear]{natbib}
% \usepackage{booktabs}
% \usepackage{graphicx}
% \graphicspath{{/home/kunz/Dokumente/Projects/Trait_DB/Invertebrate_traits/Paper/Figures/}}
% \usepackage{longtable}
% \usepackage{rotating}
% \usepackage{geometry}
% \usepackage{array}

% \definecolor{Gray}{gray}{0.9}

% \newcommand{\specialcell}[2][c]{%
%   \begin{tabular}[#1]{@{}c@{}}#2\end{tabular}}

\begin{document}
%! Trait (species) scores along segments of river

\subsection*{Re-analysis of Szöcs et al. using harmonized and aggregated grouping features}

Overall, using the harmonized grouping features lead only to slightly different results in comparison to the original analysis (Figure \ref{fig:rda_species_scores} and SI). According to the RDA of the trait composition sites with high salinity were characterized by multivoltine, ovivoparous, gill-respirating, and shredder species. Only species with the trait life cycle duration $> 1$ year fail to characterize sites with high salinization. Also, life cycle duration $<= 1$ year is not anymore characterizing sites not impacted by salinity. Like in the original analysis, transition and upstream sites from the point source are characterized by univoltine species and species that lay their eggs in an aquatic environment. We constructed also the linear models of the original analyses, using the traits on the extremes of the conductivity axis and found results similar to the original analysis (Table S\ref{stab:linear_models_new} and S\ref{stab:linear_models_edi}).

For every aggregation method compared, using at family-level aggregated traits did only slightly change the species scores compared to not aggregated traits (Figure \ref{fig:rda_species_scores}). Hence, the interpretation of the trait composition is the same as when only using harmonized grouping features. 
 
% Mention that threshold according to mahalanobis distance is almost reached for life cycle duration

% Species scores
\begin{figure}[H]
    \label{fig:rda_species_scores}
    \centering
    \includegraphics[width=16.5cm, height=10cm]{Species_scores_rda.png}
    \caption{Species scores obtained by RDA from the original analysis \cite{szocs_effects_2014}, using harmonized grouping features, and using harmonized grouping features with assigned trait affinities aggregated to family-level.}
    \label{fig:rda_species_scores}
\end{figure}

\end{document}