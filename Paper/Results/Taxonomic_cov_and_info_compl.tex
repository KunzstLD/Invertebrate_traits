% \documentclass{article}
\documentclass[../Draft_harmonization_paper.tex]{subfiles}

\begin{document}

\subsection*{Taxonomic coverage of the harmonized trait datasets}

Regarding the taxonomical coverage, the New Zealand dataset has, as expected, the smallest taxon pool (478 taxa, Table \ref{tab:tax_coverage}). By contrast, the largest taxon pool is spanned by the European trait dataset with 4496 taxa followed by the North American trait dataset that contained trait information on 3753 taxa. The Australian dataset contains 1402 taxa. The European, New Zealand, and North American datasets have most taxa on the species-level while the Australian dataset has a similar number of taxa on species and genus-level.

\begin{table}[ht]
    \centering
    \caption{Number of taxa per harmonized dataset and per taxonomic level. Numbers in parenthesis show rounded relative frequencies in percentage.} 
    \label{tab:tax_coverage}
    \begin{tabular}{lccccc}
    \toprule[.1em]
    Database & Nr. of taxa & Species & Genus & Family & Nr. aquatic insects \\ 
    \toprule[.1em]
    EU & 4496 & 3691 (82) & 647 (14) & 158 (4) & 3838 (85) \\ 
    NOA & 3753 & 2414 (64) & 1163 (31) & 176 (5) & 3305 (88) \\ 
    AUS & 1402 & 564 (40) & 578 (41) & 260 (19) & 1016 (72) \\ 
    NZ & 478 & 404 (85) & 47 (10) & 27 (6) & 443 (93) \\ 
    \bottomrule
    \end{tabular}
\end{table}


\subsection*{Completeness of trait information}

The amount of entries with available information for the selected grouping features varied strongly for the preprocessed European, North American, and Australian datasets (Table \ref{tab:trait_coverage}). By contrast, the New Zealand dataset contained complete trait information for most of the investigated grouping features (between 94 \% and 100 \%).

\begin{table}[ht]
    \centering
    \caption{Rounded percentage of entries that have 
    information for the individual grouping features
    shown per trait dataset.}
    \label{tab:trait_coverage}
    \begin{tabular}{lccccccc}
    \toprule[.1em]
    Database & Body form & Oviposition & Voltinism & Locomotion & Size & Respiration & Feeding mode \\ 
    \toprule[.1em]
    EU & 8 & 15 & 21 & 36 & 10 & 49 & 75 \\ 
    NOA & 28 & 13 & 47 & 52 & 73 & 44 & 63 \\ 
    AUS & 4 & 46 & 49 & 39 & 75 & 68 & 99 \\ 
    NZ & 100 & 94 & 100 & 99 & 100 & 100 & 99 \\ 
    \bottomrule
    \end{tabular}
\end{table}

\end{document}