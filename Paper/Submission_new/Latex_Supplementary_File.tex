\documentclass[12pt]{article}
\usepackage[utf8]{inputenc}
\usepackage{hyperref}
\usepackage{float}
\usepackage[table,xcdraw]{xcolor}
\usepackage{color, colortbl}
\usepackage{longtable}
\usepackage{booktabs}
\usepackage{graphicx}
\usepackage{multirow}
\usepackage{tikz}
\usepackage{rotating}
\usepackage{caption}
\usepackage{authblk}
\usepackage{csquotes}
\graphicspath{{Figures/}}
\usepackage{setspace}
\usepackage{rotating}
\usepackage{geometry}
\usepackage{array}
\usepackage{lscape}
\usepackage{longtable}
\usepackage{etoolbox}
\usepackage{hhline}
\usepackage{lmodern}
\usepackage[
backend = biber,
natbib,
citestyle = authoryear,
bibstyle = apa,
maxcitenames = 2, 
maxbibnames = 99, 
uniquename=false,
uniquelist=false
%sorting = none % sort by name year title
]{biblatex}
\addbibresource{Ref_invertebrate_DB.bib}

\usepackage{subfiles} % Best loaded last in the preamble

%%%% Functions and definitions %%%%%%%%%%%%%%%%%%%%%%%%%%%%%%
\definecolor{Gray}{gray}{0.9}

%% horizontal space between two columns
\setlength{\tabcolsep}{2mm}

%%%%%% New Commands %%%%%%%%%%%%%%%%%%%%%%%%%%%%%%%%%%%%%%%%%%%%%%

\newcommand{\specialcell}[2][c]{%
  \begin{tabular}[#1]{@{}c@{}}#2\end{tabular}}

% \renewcommand*{\thefootnote}{\alph{footnote}}

%%%%% Formatting options %%%%%%%%%%%%%%%%%%%%%%%%%%%%%%%%%%%%%%%%%%%%%%%
\onehalfspacing

%%%%%% Options for tables and figure counts %%%%%%%%%%%%%%%%%%%%%%%%%%%%%%%%%%%%%%%%%%%%%%%
\setcounter{table}{0}
\setcounter{figure}{0}
\renewcommand{\thetable}{S\arabic{table}}
\renewcommand{\thefigure}{S\arabic{figure}}

% \listfiles

\title{Supporting Information - Tackling discrepancies in freshwater invertebrate trait databases: Harmonising across continents and aggregating taxonomic resolution}
\author[1]{Stefan Kunz}
\author[2]{Ben J. Kefford}
\author[3]{Astrid Schmidt-Kloiber}
\author[4]{Christoph D. Matthaei}
\author[5]{Philippe Usseglio-Polatera}
\author[3]{Wolfram Graf}
\author[6]{N. LeRoy Poff}
\author[7]{Leon Metzeling}
\author[8]{Laura Twardochleb}
\author[9]{Charles P. Hawkins}
\author[1]{Ralf B. Schäfer}
\affil[1]{Institute for Environmental Sciences, University of Koblenz-Landau, Landau, Germany}
\affil[2]{Centre for Applied Water Science, Institute for Applied Ecology, University of Canberra, Canberra, Australia}
\affil[3]{Institute of Hydrobiology and Aquatic Ecosystem Management, University of Natural Resources and Life Sciences Vienna (BOKU), Vienna, Austria}
\affil[4]{Department of Zoology, University of Otago, Dunedin, New Zealand}
\affil[5]{University of Lorraine, CNRS, LIEC, Metz, France}
\affil[6]{Department of Biology, Colorado State University, Fort Collins, USA}
\affil[7]{Environment Protection Authority Victoria, Applied Sciences Division, Macleod, Australia}
\affil[8]{Department of Fisheries and Wildlife, Michigan State University, East Lansing, USA}
\affil[9]{Department of Watershed Sciences, National Aquatic Monitoring Center, and the Ecology Center, Utah State University, Logan, USA}
\date{}


\begin{document}
\maketitle

\begin{landscape}
    \begin{longtable}{m{2cm}|m{3.25cm}|m{3.2cm}|m{3.1cm}|m{2.9cm}|m{3.4cm}|m{2.4cm}}
        \caption{Comparison of trait definition differences between invertebrate trait databases. Only traits that are differently described across databases are listed. The definition is quoted if it enables differences to be identified, otherwise the differences are described. Bullet points indicate when several traits have been used to describe a morphological or behavioural property. The hyphen indicates a missing trait. Reproduction was captured in multiple grouping features per database. Hence, differences for reproduction have been described in the paper. Body form traits are not different between databases, except that the Vieira database contains the trait Bluff (blocky) which does not appear in the other databases.}
        \label{stab:trait_definitions}
        \endfirsthead
        \toprule[.1em]
        Trait & \specialcell{Freshwater- \\ ecology.info} & Tachet & CONUS & Vieira & Australia & \specialcell{New \\ Zealand} \\
        \toprule[.1em]
        Feeding shredder & 
        "Feed from fallen leaves, plant tissues, CPOM" & 
        "Eat coarse detritus, plants or \textit{animal material}" & 
        \begin{itemize}
            \item "Shred decomposing vascular plant tissue"
            \item Trait herbivore includes among others insect that shred \textit{living aquatic plants} 
        \end{itemize} & 
        Shredder & 
        \begin{itemize}
            \item Detrivore$^{\dagger}$
            \item Trait herbivore includes among others the trait shredder
        \end{itemize} & 
        Shredders
        \\ 
        \midrule
        Feeding predator & 
        "Eating from prey" & 
        \begin{itemize}
            \item Carvers, engulfers \& swallowers
            \item Piercers (plants \& animals) are an additional trait
        \end{itemize} & % Notes: Tachet -> Piercer (plants & animals)
        Engulfers ("ingest prey whole or in parts") \& 
        piercers ("prey tissues and suck fluids") & 
        Predator &
        Piercer \& engulfer &
        Predator
        \\ 
        \midrule
        Feeding filter-feeder & 
        \begin{itemize}
            \item Active filter feeders
            \item Passive filter feeders
        \end{itemize} &
        No distinction between active and passive &
        No distinction between active and passive &
        No distinction between active and passive &
        No distinction between active and passive &
        No distinction between active and passive
        \\
        \toprule[.1em]
        Semivoltine & 
        "One generation in two years" & 
        "Life cycle lasts \textit{at least} two years" & 
        "$< 1$ generation per year" & 
        "$< 1$ generation per year" & 
        "$< 1$ generation per year" & 
        "$< 1$ reproductive cycle per year"
        \\
        \midrule
        Multi\-voltine & 
        "\textit{Three} or more generations per year"$^{\ddagger}$ & 
        "Able to complete \textit{at least} two successive generations per year" &
        "$> 1$ generations per year" &
        "$> 1$ generations per year" & 
        \begin{itemize}
            \item 1-2 generations per year
            \item bi - or multivoltine
            \item up to 5 generations per year
            \item up to 10 generations per year
        \end{itemize}
        & 
        "$> 1$ reproductive cycles per year"
        \\
        \toprule[.1em]
        Locomotion swimming & 
        \begin{itemize}
            \item Passive movement like floating or drifting (trait swimming/scating)
            \item Active movement (trait swimming/diving)
        \end{itemize}. &
        \begin{itemize}
            \item Surface swimmers (over and under the water surface)
            \item Full water swimmers (e.g. Baetidae).
        \end{itemize} & 
        "Adapted for "fishlike" swimming" & 
        Swimmer & 
        \begin{itemize}
            \item Swimmer
            \item Skater
        \end{itemize} & 
        Swimmers (water column)
        \\
        \midrule
        Locomotion burrowing & 
        "Burrowing in \textit{soft} substrates or boring in \textit{hard} substrates" & 
        \begin{itemize}
            \item Burrowing "within the first centimeters of the benthic fine sediment"
            \item Interstitial (endobenthic)
        \end{itemize} & 
        "Inhabiting \textit{fine} sediment of streams and lakes" &
        Burrower & 
        "Moving deep into the substrate and thus avoiding flow" &
        Burrowers (infauna)
        \\
        \midrule
        Locomotion sprawling \& walking & 
        "Sprawling or walking actively with legs, pseudopods or on a mucus" &
        - & 
        Sprawling: "inhabiting the surface of floating leaves of vascular hydrophytes or fine sediments" & 
        Sprawler &
        - & 
        - \\
        \midrule
        Locomotion crawling & 
        - &
        "Crawling over the bottom substrate" & 
        Defined as crawling on the surface of floating leaves or fine sediments on the bottom & 
        - & 
        \begin{itemize}
            \item Crawler
            \item Sprawler
            \item Climber
            \item Clinger
        \end{itemize} &
        Crawlers (epibenthic) \\
        \midrule
        Locomotion sessil & 
        Does not distinguish temporarily and permanently attached & 
        \begin{itemize}
            \item Temporarily attached
            \item Permanently attached
        \end{itemize} & 
        Does not distinguish temporarily and permanently attached & 
        Does not distinguish temporarily and permanently attached & 
        \begin{itemize}
            \item Temporarily attached
            \item Permanently attached
        \end{itemize} & 
        Does not distinguish temporarily and permanently attached \\
        \toprule[.1em]
        Respiration plastron \& spiracle & 
        Plastron and spiracle (aerial) are two separate traits & 
        Definition includes respiration using air stores of aquatic plants & 
        Plastron and spiracle combined into one trait & 
        \begin{itemize}
            \item Spiracular gills
            \item Plastron
            \item Atmosph. breathers
            \item Plant breathers
        \end{itemize} &
        \begin{itemize}
            \item Plastron and spiracle (termed aerial) occur as separate and combined traits
            \item Air (plants)
            \item Atmospheric
            \item Functional spiracles
        \end{itemize} &
        \begin{itemize}
            \item Plastron
            \item Spiracle (termed aerial)
        \end{itemize} 
        \\
        \toprule[.1em]
        Body size small & 
        - &
        \multirow{3}{*}{\specialcell{Multiple size \\ classifications$^{\P}$}} & 
        $<$ 9 mm & 
        $<$ 9 mm & 
        $<$ 9 mm $^{\dagger \mathsection}$ &
        \multirow{3}{*}{\specialcell{Multiple size \\ classifications$^{\star}$}}  
        \\
        \cline{1-2}
        \cline{4-6}
        Body size medium & 
        - &
        &
        9 - 16 mm & 
        9 - 16 mm & 
        9 - 16 mm &
        \\
        \cline{1-2}
        \cline{4-6}
        Body size large & 
        - &
        &
        $>$ 16 mm &
        $>$ 16 mm &
        $>$ 16 mm &
        \\
        \bottomrule
    \end{longtable}
    \begin{minipage}{\linewidth}{\fontsize{8}{10}\selectfont
        $\dagger$ Traits from Botwe et al.
        \newline
        $\ddagger$ Contains also bivoltine (two generations per year), trivoltine (three generations per year) and flexible.
        \newline
        $\mathsection$ Contains a size trait with numeric size values. Contains also traits classifying size like Tachet and like the North American trait databases. 
        \newline
        $\P$ Size classifications: \textit{$<=0.25$ cm, $> 0.25-0.5$ cm, $0.5-1$ cm, $1-2$ cm, $2-4$ cm, $4-8$ cm, $> 8$ cm}. No distinction into small, medium and large.
        \newline
        $\star$ Size classifications: \textit{$> 0.25-0.5$ cm, $0.5-1$ cm, $1-2$ cm, $2-4$ cm, $4-8$ cm}. No distinction into small, medium and large.
        }
    \end{minipage}
\end{landscape}

\newpage

\subsection*{Comparing aggregation methods}

\subsubsection*{Comparison of family-level aggregated traits with family-level assigned traits}

\begin{figure}[H]
    \centering
    \includegraphics[width=15cm, height=12cm]{Deviances_trait_agg_pyne.png}
    \caption{Cases (factor combination of investigated families and traits) where differences occurred between aggregated traits and expert assigned traits at family level for the North American dataset. Violin plots - mirrored density plots - show the density of the absolute trait affinity differences for the grouping features locomotion, respiration, and body size. For more details see Figure 2.}%\ref{fig:diff_aggr_traits_combined}
    \label{fig:diff_aggr_traits_pyne}
\end{figure}

\subsubsection*{Effect of phylogeny and trait variability on aggregation outcomes}


To examine the influence of phylogeny and trait variability on the outcomes of the different trait aggregation methods, we created three hypothetical scenarios. We simulated three different families, each containing 25 total species but with different phylogenetic structures. The three families consisted of (1) a family with an equal number of genera and species (five genera each with five species each), denoted as \textit{sim\_base}; (2) a family in which one genus had a much larger number of species than the other four genera (1 genus with 13 species, 4 genera with 3 species each), denoted as \textit{sim\_extreme}; (3) a family in which all genera had a different number of species (8, 2, 7, 3, 5), denoted as \textit{sim\_variation}. We assigned a hypothetical grouping feature with three traits (T1, T2, and T3) to each scenario. We then simulated the 25 affinities, one per species, for each trait, by sampling from a truncated normal distribution bound by 0 and 1 and with a mean value of 0.5. To simulate different levels of trait variability, we repeated the sampling 100 times for each of the 5 standard deviations (0.2, 0.4, 0.6, 0.8, and 1), resulting in 12,500 simulated trait affinities for each simulated trait ($25$ species per family $\times$ 5 levels of trait variability $\times$ 100 replicates). We converted simulated trait affinities to proportions, as was done during trait database processing, and assigned the 12,500 simulated affinities per trait to each of the three family scenarios. We then applied the five trait aggregation methods described above to each simulated dataset. We compared the resulting ranges of aggregated trait affinities between levels of trait variability and phylogenetic scenario as well as the differences in trait affinities obtained by each aggregation method.


\paragraph*{Results: Comparison of aggregation methods with varying phylogenies and trait variability}
\hfill
\\

The simulations showed that both phylogenetic structure and trait variability affected aggregated trait affinities, although only to a small degree, and that aggregation method mattered in terms of the ranges of trait affinities produced over simulation replicates. The effect of phylogenetic structure differed across aggregation methods, but, as we expected, the range of trait affinities increased with increasing trait variability for all aggregation methods (Figure \ref{fig:overview_sim_results}). 

Phylogenetic structure appeared to influence the outcomes of the different aggregation methods. For the \textit{sim\_base} scenario (equal numbers of genera and species), the mean aggregation methods yielded similar ranges of aggregated trait affinities within each level of trait variability, and the median aggregation methods consistently produced greater ranges of aggregated trait affinities than the other methods. The largest ranges were produced by the \textit{stepwise\_agg \textsubscript{median}} method. For the more complex phylogenetic structures, \textit{sim\_extreme} (one genus with a much larger number of species than the other four) and \textit{sim\_variation} (all genera with a different number of species), the \textit{stepwise\_agg \textsubscript{median}} method still produced the largest ranges of trait affinities for most levels of trait variability (and \textit{direct\_agg\textsubscript{mean}} produced the narrowest ranges for most levels of trait variability), but there was much less consistency in the ranges of trait affinities for all aggregation methods (i.e., the ranges were different among all aggregation methods).

Although trait aggregation methods were affected by phylogenetic structure and trait variability to some degree, in most simulated datasets, the different aggregation methods resulted in similar trait affinities. Only 1.4 \%, or 213 out of 15.000 total comparisons (3 scenarios $\times$ 5 levels of trait variability $\times$ 10 unique comparisons of trait aggregation methods $\times$ 100 replicates) showed a difference equal or greater than an absolute trait affinity of 0.1. Most  (83.5 \%) of these differences occurred in the \textit{sim\_extreme} scenario and were found between the aggregation methods \textit{direct\_agg\textsubscript{mean}} and \textit{stepwise\_agg\textsubscript{median}}, \textit{direct\_agg\textsubscript{median}} and \textit{stepwise\_agg\textsubscript{median}}, and \textit{stepwise\_agg\textsubscript{median}} and \textit{weighted\_agg} (Figure \ref{fig:sim_indv_runs}).



\begin{figure}[H]
    \centering
    \includegraphics[width=16.5cm, height=10cm]{Overview_sim_results.png}
    \caption{Ranges of aggregated trait affinities for the three examples of phylogenetic structures and simulated levels of trait variability. Shown are the results only for one simulated trait (T1). Similar results were obtained for the other simulated traits (Figure \ref{fig:overview_sim_results_T2_T3}). Boxplots depict results for 100 replicated simulations of each trait aggregation method. The boxplot depicts the median and encompasses the 25\textsuperscript{th} and 75\textsuperscript{th} percentile. Horizontal black lines depict the median. Whiskers extend to the largest and smallest value respectively no further than 1.5 $\times$ the inter-quartile range. Outliers beyond the end of the whiskers are plotted as grey dots. Trait aggregation methods are in order of least to greatest produced ranges to improve visual inspection.}
    \label{fig:overview_sim_results}
  \end{figure}

  \begin{figure}[H]
    \centering
    \includegraphics[width=16.5cm, height=10cm]{Overview_sim_results_T2_T3.png}
    \caption{Ranges of aggregated trait affinities for the three examples of taxonomic hierarchies and simulated levels of trait variability. Shown are the results for the simulated traits T2 and T3. Boxplots depict results for 100 replicated simulations of each trait aggregation method. Trait aggregation methods are in order of least to greatest produced ranges to improve visual inspection. For more details see Figure 3.}%\ref{fig:overview_sim_results}
    \label{fig:overview_sim_results_T2_T3}
  \end{figure}
  
  \begin{figure}[H]
    \centering
    \includegraphics[width=16.5cm, height=10cm]{Diffs_indiv_runs_sim.png}
    \caption{Comparison between the aggregated trait affinities produced by the different trait aggregation methods for every simulated dataset across all 3 simulated traits (5,000 comparisons per scenario). Results are shown for different levels of trait variability which is represented by the standard deviation (SD) of the simulated traits. Dots depict comparisons where absolute differences between aggregated trait affinities were greater than 0.1. Overall, there are 10 possible unique comparisons of which 7 produced absolute differences greater than 0.1. \newline
    Comparisons: \newline
    1) \textit{direct\_agg\textsubscript{median}} - \textit{stepwise\_agg\textsubscript{mean}} \newline
    2) \textit{direct\_agg\textsubscript{median}} - \textit{weighted\_agg} \newline
    3) \textit{stepwise\_agg\textsubscript{mean}} - \textit{stepwise\_agg\textsubscript{median}} \newline
    4) \textit{stepwise\_agg\textsubscript{mean}} - \textit{weighted\_agg} \newline  
    5) \textit{direct\_agg\textsubscript{mean}} - \textit{stepwise\_agg\textsubscript{median}} \newline
    6) \textit{direct\_agg\textsubscript{median}} - \textit{stepwise\_agg\textsubscript{median}} \newline
    7) \textit{stepwise\_agg\textsubscript{median}} - \textit{weighted\_agg} \newline
    }
    \label{fig:sim_indv_runs}
  \end{figure}

\paragraph*{Discussion: Comparison of aggregation methods with varying phylogenies and trait variability}
\hfill
\\

We expected that, in addition to averaging measures, different weighting approaches to aggregation (i.e., equal weight to each taxon, equal weight to each genus, or weighted by number of species) would affect affinities for families with varying phylogenetic structure. However, we found through simulations that trait variability had a greater effect on the range of affinities (i.e., greater ranges of affinities at higher levels of variance) and on the differences in affinities between weighting approaches. In the simulation, taxonomic structural unevenness appeared to produce some variation in affinities among weighting approaches, especially for a family where one genus had a much larger number of species than its other genera. The comparison between aggregated and assigned traits showed, that the number of differing cases and the mean absolute differences in trait affinities, as well as the distributions of absolute trait affinity differences to assigned traits, were similar across the mean aggregation and across the median aggregation methods, which suggests a small influence of the weighting approach on the aggregation outcomes. The minor impact of the weighting approaches on trait aggregation might be explained by the fact that a considerable portion of taxa had low numbers of genera or species. Of the taxa that were compared from the North American trait dataset, 14 \% were identified at family and 62 \% at genus level, 52 \% comprised five or fewer genera, and 13 \% contained just one genus (Figure \ref{fig:tax_hierarchy_NOA}). In the Australian dataset, 21 \% of the compared taxa were identified at family, 40 \% at genus, and 39 \% at species level, 68 \% of the taxa contained five or fewer genera, and 40 \% just one genus (Figure \ref{fig:tax_hierarchy_AUS}). Hence, these results could change when more species-level trait information becomes available. Our findings show that the choice of aggregation method may matter less when phylogenetic structure is fairly even and traits are less variable, but the \textit{stepwise\_agg\textsubscript{median}} method tended to produce the widest range in affinities, especially with high taxonomic unevenness and high trait variability. 


\subsubsection*{Taxonomic hierarchy in the trait datasets used for comparisons with assigned traits at family level}
\label{sec:taxonomic_hierarchy}

\begin{figure}[H]
    \centering
    \includegraphics[width=16.5cm, height=10cm]{taxonomic_hierarchy_NOA.png}
    \caption{Number of genera per family and species per genus for those families of the North American trait dataset that have been compared with assigned traits at family level. For better visual display only families with more than 15 genera are displayed.}
    \label{fig:tax_hierarchy_NOA}
\end{figure}

\begin{figure}[H]
    \centering
    \includegraphics[width=16.5cm, height=10cm]{taxonomic_hierarchy_AUS.png}
    \caption{Number of genera per family and species per genus for the Australian trait dataset. For better visual display only families with more than 7 genera are displayed.}
    \label{fig:tax_hierarchy_AUS}
\end{figure}

\newpage

\subsection*{Effects of harmonisation and trait aggregation on inferences regarding trait-environment relationships}

\begin{table}[H]
    \centering
    \caption{Mean, median and standard deviation of the affinities of traits that were responsive to the salinity gradient in the original study but not in the re-analysis using the harmonised European trait dataset.} 
    \label{tab:SI_resp_traits_summary_stats}
    \begin{tabular}{l|l|c|c|c|c}
    \toprule[.1em]
    Type & Trait & Mean & Median & SD & Responsive? \\ 
    \toprule[.1em]
    Stepw\_median & Shredder & 0.20 & 0.14 & 0.25 & No \\ 
      Stepw\_mean & Shredder & 0.18 & 0.12 & 0.22 & No\\ 
      Direct\_median & Shredder & 0.21 & 0.14 & 0.25 & No\\ 
      Direct\_mean & Shredder & 0.19 & 0.14 & 0.22 & No\\ 
      Weighted & Shredder & 0.19 & 0.14 & 0.22 & No \\ 
      Harmonised; not\_aggregated & Shredder & 0.18 & 0.12 & 0.24 & No \\ 
      Original & Shredder & 0.25 & 0.14 & 0.32 & Yes\\ 
      \midrule
      Stepw\_median & Gills & 0.30 & 0.27 & 0.32 & Yes\\ 
      Stepw\_mean & Gills & 0.29 & 0.22 & 0.32 & Yes\\ 
      Direct\_median & Gills & 0.30 & 0.30 & 0.32 & Yes\\ 
      Direct\_mean & Gills & 0.30 & 0.30 & 0.32 & Yes\\ 
      Weighted & Gills & 0.30 & 0.30 & 0.32 & Yes\\ 
      Harmonised; not\_aggregated & Gills & 0.30 & 0.25 & 0.32 & No \\ 
      Original & Gills & 0.28 & 0.00 & 0.33 & Yes \\ 
      \midrule
      Stepw\_median & Short life cycle & 0.64 & 0.75 & 0.39 & No \\ 
      Stepw\_mean & Short life cycle & 0.64 & 0.79 & 0.39 & No \\ 
      Direct\_median & Short life cycle & 0.67 & 0.75 & 0.37 & Yes \\ 
      Direct\_mean & Short life cycle & 0.67 & 0.79 & 0.38 & Yes \\ 
      Weighted & Short life cycle & 0.67 & 0.79 & 0.38 & Yes\\ 
      Harmonised; not\_aggregated & Short life cycle & 0.64 & 0.75 & 0.40 & Yes \\ 
      Original & Short life cycle & 0.64 & 0.75 & 0.40 & Yes \\ 
      \midrule
      Stepw\_median & Long life cylce & 0.36 & 0.25 & 0.39 & No \\ 
      Stepw\_mean & Long life cylce & 0.36 & 0.21 & 0.39 & No \\
      Direct\_median & Long life cylce & 0.33 & 0.25 & 0.37 & Yes\\ 
      Direct\_mean & Long life cylce & 0.33 & 0.21 & 0.38 & Yes \\ 
      Weighted & Long life cylce & 0.33 & 0.21 & 0.38 & Yes \\ 
      Harmonised; not\_aggregated & Long life cylce & 0.36 & 0.25 & 0.40 & Yes \\ 
      Original & Long life cylce & 0.36 & 0.25 & 0.40 & Yes\\ 
    \bottomrule
    \end{tabular}
\end{table}

\begin{figure}[H]
    \centering
    \includegraphics[width=18.5cm, height=16cm]{boxplot_scores_combined_REMAIN_SI.png}
    \caption{RDA of traits constrained by electric conductivity for the data aggregated with \textit{direct\_agg \textsubscript{mean}}, \textit{stepwise\_agg \textsubscript{mean}}, and \textit{weighted\_agg}. Shown are boxplots of the site scores along the conductivity axis. The rug on the right side of each plot indicates species scores of the traits on the conductivity axis. For more details see Figure 6. %\ref{fig:boxplot_scores_on_constrained_axis}. 
    Abbreviations: lcd, life cycle duration; nr.cy, potential number of cycles per year.}
    \label{fig:boxplots_scores_on_constrained_axis_REMAIN}
\end{figure}

\subsubsection*{Fourth corner analysis}

We selected the ecologically relevant traits identified by \citet{szocs_effects_2014} that increased (shredder, life cycle duration $ > 1 $ year, gills, bi/multivoltine, ovoviviparity) or decreased (univoltine, eggs in clutches/cemented or fixed, life cycle duration $ <= 1$ year) with increasing salinity and used these in the fourth corner analysis. The selected traits have also reacted to salinity in other studies (but see the discussion in \citet{szocs_effects_2014}). 

For some sites, environmental and abundance data have been sampled multiple times per year. \citet{szocs_effects_2014} used the full data and did not consider repeated sampling. We applied two analysis: one where we did not consider multiple samplings. And second where we selected, for the sites with multiple samplings, only those sampling events with the highest macroinvertebrate abundances per year and accordingly only used the conductivity data for these sampling events (this was not done for the RDA approach).Fourth corner analysis was carried out in R with the \textit{fourthcorner()} function from the ade4 package.

Fourth corner analysis detected fewer trait-conductivity relationships than the RDA approach. With the original dataset significant relationships to conductivity were found for the traits shredder and ovoviviparity. In three out of the six aggregated and harmonised datasets (\textit{direct\_agg\textsubscript{mean}}, \textit{weighted\_agg}, \textit{direct\_agg\textsubscript{median}}) four significant relationships (for the traits univoltinism, bi/multivoltine, ovoviviparity, and aquatic oviposition) were detected. In the remaining three datasets (\textit{stepwise\_agg\textsubscript{median}}, \textit{stepwise\_agg\textsubscript{mean}}, and harmonised (not aggregated)) two significant relationships for the traits univoltinism and bi/multivoltinism were detected (Table \ref{stab:fc_selected_reduced}). Results were only slightly different when not considering multiple samplings (original dataset: two significant trait-conductivity relationship (bi-multivoltine, ovoviviparity), other datasets two to four, Table \ref{stab:fc_selected_full}). For all significant trait-conductivity relationships, the classification into characteristic traits for downstream and upstream sites was consistent with the results of the RDA analysis.

\begin{longtable}[H]{m{2.6cm}|m{7.3cm}|m{1cm}|m{1.4cm}|m{1.5cm}|m{1.5cm}}
    \caption{Results of the fourth corner analysis, after accounting for multiple samplings per year, with the original dataset, harmonised, and aggregated trait datasets.}
    \label{stab:fc_selected_reduced}
    \endfirsthead
    \toprule[.1em]
    Dataset & Test & Obs & Std.Obs & Pvalue & Pvalue.adj \\ 
    \toprule[.1em]
       stepw\_median & cond / feed\_shredder & 0.45 & 1.86  & 0.04 & 0.16 \\ 
       stepw\_median & cond / resp\_gil & 0.47 & 1.87  & 0.05 & 0.16 \\ 
       stepw\_median & cond / volt\_uni & -0.57 & -2.29  & 0.00 & 0.03 \\ 
       stepw\_median & cond / volt\_bi\_multi & 0.58 & 2.35  & 0.00 & 0.02 \\ 
       stepw\_median & cond / ovip\_aqu & -0.53 & -2.14  & 0.01 & 0.06 \\ 
       stepw\_median & cond / ovip\_ovo & 0.54 & 2.15  & 0.01 & 0.06 \\ 
       stepw\_median & cond / Life cycle duration $<=$ 1 year & -0.32 & -1.27  & 0.24 & 0.49 \\ 
       stepw\_median & cond / Life cycle duration $>$ 1 year & 0.32 & 1.27  & 0.24 & 0.49 \\ 
       stepw\_mean & cond / feed\_shredder & 0.46 & 1.96  & 0.03 & 0.11 \\ 
       stepw\_mean & cond / resp\_gil & 0.47 & 1.75  & 0.06 & 0.18 \\ 
       stepw\_mean & cond / volt\_uni & -0.58 & -2.45  & 0.01 & 0.04 \\ 
       stepw\_mean & cond / volt\_bi\_multi & 0.59 & 2.50  & 0.00 & 0.03 \\ 
       stepw\_mean & cond / ovip\_aqu & -0.52 & -2.23  & 0.01 & 0.06 \\ 
       stepw\_mean & cond / ovip\_ovo & 0.52 & 2.23  & 0.01 & 0.06 \\ 
       stepw\_mean & cond / Life cycle duration $<=$ 1 year & -0.33 & -1.30  & 0.23 & 0.47 \\ 
       stepw\_mean & cond / Life cycle duration $>$ 1 year & 0.33 & 1.30  & 0.23 & 0.47 \\ 
       direct\_median & cond / feed\_shredder & 0.45 & 1.89  & 0.04 & 0.15 \\ 
       direct\_median & cond / resp\_gil & 0.47 & 1.88  & 0.04 & 0.15 \\ 
       direct\_median & cond / volt\_uni & -0.58 & -2.24  & 0.00 & 0.02 \\ 
       direct\_median & cond / volt\_bi\_multi & 0.58 & 2.33  & 0.00 & 0.02 \\ 
       direct\_median & cond / ovip\_aqu & -0.53 & -2.14  & 0.01 & 0.05 \\ 
       direct\_median & cond / ovip\_ovo & 0.54 & 2.15  & 0.01 & 0.05 \\ 
       direct\_median & cond / Life cycle duration $<=$ 1 year & -0.34 & -1.35  & 0.21 & 0.43 \\ 
       direct\_median & cond / Life cycle duration $>$ 1 year & 0.34 & 1.35  & 0.21 & 0.43 \\ 
       direct\_mean & cond / feed\_shredder & 0.46 & 1.83  & 0.05 & 0.19 \\ 
       direct\_mean & cond / resp\_gil & 0.47 & 1.82  & 0.05 & 0.19 \\ 
       direct\_mean & cond / volt\_uni & -0.58 & -2.49  & 0.00 & 0.02 \\ 
       direct\_mean & cond / volt\_bi\_multi & 0.59 & 2.54  & 0.00 & 0.02 \\ 
       direct\_mean & cond / ovip\_aqu & -0.52 & -2.26  & 0.01 & 0.05 \\ 
       direct\_mean & cond / ovip\_ovo & 0.52 & 2.43  & 0.01 & 0.05 \\ 
       direct\_mean & cond / Life cycle duration $<=$ 1 year & -0.34 & -1.37  & 0.18 & 0.37 \\ 
       direct\_mean & cond / Life cycle duration $>$ 1 year & 0.34 & 1.37  & 0.18 & 0.37 \\ 
       weighted & cond / feed\_shredder & 0.46 & 1.97  & 0.03 & 0.11 \\ 
       weighted & cond / resp\_gil & 0.47 & 1.92  & 0.04 & 0.11 \\ 
       weighted & cond / volt\_uni & -0.58 & -2.51  & 0.00 & 0.02 \\ 
       weighted & cond / volt\_bi\_multi & 0.59 & 2.56  & 0.00 & 0.02 \\ 
       weighted & cond / ovip\_aqu & -0.52 & -2.13  & 0.01 & 0.04 \\ 
       weighted & cond / ovip\_ovo & 0.52 & 2.35  & 0.01 & 0.04 \\ 
       weighted & cond / Life cycle duration $<=$ 1 year & -0.34 & -1.38  & 0.20 & 0.41 \\ 
       weighted & cond / Life cycle duration $>$ 1 year & 0.34 & 1.38  & 0.20 & 0.41 \\ 
       not\_aggregated & cond / feed\_shredder & 0.51 & 2.08  & 0.01 & 0.08 \\ 
       not\_aggregated & cond / resp\_gil & 0.46 & 1.72  & 0.08 & 0.40 \\ 
       not\_aggregated & cond / volt\_uni & -0.64 & -2.76  & 0.00 & 0.02 \\ 
       not\_aggregated & cond / volt\_bi\_multi & 0.65 & 2.81  & 0.00 & 0.02 \\ 
       not\_aggregated & cond / ovip\_aqu & -0.26 & -1.22  & 0.27 & 0.80 \\ 
       not\_aggregated & cond / ovip\_ovo & 0.27 & 1.39  & 0.20 & 0.80 \\ 
       not\_aggregated & cond / Life cycle duration $<=$ 1 year & -0.33 & -1.26   & 0.24 & 0.80 \\ 
       not\_aggregated & cond / Life cycle duration $>$ 1 year & 0.33 & 1.26  & 0.24 & 0.80 \\ 
       original & cond / Life cycle duration $<=$ 1 year & -0.33 & -1.31  & 0.22 & 0.65 \\ 
       original & cond / Life cycle duration $>$ 1 year & 0.33 & 1.31  & 0.22 & 0.65 \\ 
       original & cond / Potential number of cycles per year $1$ & -0.48 & -1.91  & 0.04 & 0.18 \\ 
       original & cond / Potential number of cycles per year $> 1$  & 0.49 & 1.95  & 0.03 & 0.16 \\ 
       original & cond / Reproduction ovoviviparity & 0.58 & 2.43  & 0.01 & 0.04 \\ 
       original & cond / Reproduction clutches, cemented or fixed & -0.33 & -1.27  & 0.24 & 0.65 \\ 
       original & cond / Respiration gill & 0.47 & 1.87  & 0.04 & 0.18 \\ 
       original & cond / Feeding habits shredder & 0.54 & 2.51  & 0.01 & 0.04 \\ 
      \bottomrule
\end{longtable}


\begin{longtable}[H]{m{2.6cm}|m{7.3cm}|m{1cm}|m{1.4cm}|m{1.5cm}|m{1.5cm}}
    \caption{Results of the fourth corner analysis without accounting for multiple samplings per year, with the original dataset, harmonised, and aggregated trait datasets.}
    \label{stab:fc_selected_full}
    \endfirsthead
    \toprule[.1em]
    Dataset & Test & Obs & Std.Obs & Pvalue & Pvalue.adj \\ 
    \toprule[.1em]
    stepw\_median & cond / feed\_shredder & 0.38 & 1.86 & 0.05 & 0.19 \\ 
      stepw\_median & cond / resp\_gil & 0.38 & 1.80 & 0.06 & 0.19 \\ 
      stepw\_median & cond / volt\_uni & -0.54 & -2.43 & 0.00 & 0.03 \\ 
      stepw\_median & cond / volt\_bi\_multi & 0.54 & 2.50 & 0.00 & 0.02 \\ 
      stepw\_median & cond / ovip\_aqu & -0.46 & -2.34 & 0.01 & 0.04 \\ 
      stepw\_median & cond / ovip\_ovo & 0.46 & 2.57 & 0.01 & 0.04 \\ 
      stepw\_median & cond / Life cycle duration $<=$ 1 year & -0.28 & -1.29 & 0.22 & 0.45 \\ 
      stepw\_median & cond / Life cycle duration $>$ 1 year & 0.28 & 1.29 & 0.22 & 0.45 \\ 
      stepw\_mean & cond / feed\_shredder & 0.39 & 1.93 & 0.03 & 0.12 \\ 
      stepw\_mean & cond / resp\_gil & 0.39 & 1.81 & 0.06 & 0.17 \\ 
      stepw\_mean & cond / volt\_uni & -0.54 & -3.65 & 0.00 & 0.01 \\ 
      stepw\_mean & cond / volt\_bi\_multi & 0.55 & 3.71 & 0.00 & 0.01 \\ 
      stepw\_mean & cond / ovip\_aqu & -0.44 & -2.26 & 0.01 & 0.05 \\ 
      stepw\_mean & cond / ovip\_ovo & 0.45 & 2.50 & 0.00 & 0.02 \\ 
      stepw\_mean & cond / Life cycle duration $<=$ 1 year & -0.28 & -1.34 & 0.20 & 0.39 \\ 
      stepw\_mean & cond / Life cycle duration $>$ 1 year & 0.28 & 1.34 & 0.20 & 0.39 \\ 
      direct\_median & cond / feed\_shredder & 0.37 & 1.89 & 0.03 & 0.13 \\ 
      direct\_median & cond / resp\_gil & 0.38 & 1.82 & 0.05 & 0.16 \\ 
      direct\_median & cond / volt\_uni & -0.54 & -3.79 & 0.00 & 0.01 \\ 
      direct\_median & cond / volt\_bi\_multi & 0.55 & 3.85 & 0.00 & 0.01 \\ 
      direct\_median & cond / ovip\_aqu & -0.46 & -2.37 & 0.01 & 0.05 \\ 
      direct\_median & cond / ovip\_ovo & 0.46 & 2.56 & 0.01 & 0.05 \\ 
      direct\_median & cond / Life cycle duration $<=$ 1 year & -0.30 & -1.41 & 0.18 & 0.35 \\ 
      direct\_median & cond / Life cycle duration $>$ 1 year & 0.30 & 1.41 & 0.18 & 0.35 \\ 
      direct\_mean & cond / feed\_shredder & 0.39 & 1.94 & 0.03 & 0.11 \\ 
      direct\_mean & cond / resp\_gil & 0.38 & 1.79 & 0.06 & 0.17 \\ 
      direct\_mean & cond / volt\_uni & -0.55 & -3.54 & 0.00 & 0.01 \\ 
      direct\_mean & cond / volt\_bi\_multi & 0.55 & 3.60 & 0.00 & 0.01 \\ 
      direct\_mean & cond / ovip\_aqu & -0.44 & -2.25 & 0.01 & 0.03 \\ 
      direct\_mean & cond / ovip\_ovo & 0.45 & 2.52 & 0.00 & 0.02 \\ 
      direct\_mean & cond / Life cycle duration $<=$ 1 year & -0.30 & -1.49 & 0.15 & 0.30 \\ 
      direct\_mean & cond / Life cycle duration $>$ 1 year & 0.30 & 1.49 & 0.15 & 0.30 \\ 
      weighted & cond / feed\_shredder & 0.39 & 1.84 & 0.05 & 0.19 \\ 
      weighted & cond / resp\_gil & 0.38 & 1.81 & 0.06 & 0.19 \\ 
      weighted & cond / volt\_uni & -0.55 & -3.43 & 0.00 & 0.01 \\ 
      weighted & cond / volt\_bi\_multi & 0.55 & 3.49 & 0.00 & 0.01 \\ 
      weighted & cond / ovip\_aqu & -0.44 & -2.24 & 0.01 & 0.05 \\ 
      weighted & cond / ovip\_ovo & 0.45 & 2.44 & 0.01 & 0.05 \\ 
      weighted & cond / Life cycle duration $<=$ 1 year & -0.30 & -1.42 & 0.17 & 0.34 \\ 
      weighted & cond / Life cycle duration $>$ 1 year & 0.30 & 1.42 & 0.17 & 0.34 \\ 
      not\_aggregated & cond / feed\_shredder & 0.43 & 2.07 & 0.01 & 0.09 \\ 
      not\_aggregated & cond / resp\_gil & 0.38 & 1.77 & 0.06 & 0.30 \\ 
      not\_aggregated & cond / volt\_uni & -0.57 & -3.90 & 0.00 & 0.01 \\ 
      not\_aggregated & cond / volt\_bi\_multi & 0.58 & 3.96 & 0.00 & 0.01 \\ 
      not\_aggregated & cond / ovip\_aqu & -0.25 & -1.28 & 0.22 & 0.62 \\ 
      not\_aggregated & cond / ovip\_ovo & 0.25 & 1.50 & 0.15 & 0.62 \\ 
      not\_aggregated & cond / Life cycle duration $<=$ 1 year & -0.29 & -1.37 & 0.18 & 0.62 \\ 
      not\_aggregated & cond / Life cycle duration $>$ 1 year & 0.29 & 1.37 & 0.18 & 0.62 \\ 
      original & cond / Life cycle duration $<=$ 1 year & -0.29 & -1.42 & 0.17 & 0.50 \\ 
      original & cond / Life cycle duration $>$ 1 year & 0.29 & 1.42 & 0.17 & 0.50 \\ 
      original & cond / Potential number of cycles per year $1$ & -0.47 & -2.11 & 0.02 & 0.10 \\ 
      original & cond / Potential number of cycles per year $> 1$ 1 & 0.47 & 2.19 & 0.01 & 0.05 \\ 
      original & cond / Reproduction ovoviviparity & 0.51 & 3.06 & 0.00 & 0.01 \\ 
      original & cond / Reproduction clutches, cemented or fixed & -0.34 & -1.49 & 0.17 & 0.50 \\ 
      original & cond / Respiration gill & 0.38 & 1.80 & 0.06 & 0.25 \\ 
      original & cond / Feeding habits shredder & 0.45 & 2.19 & 0.01 & 0.06 \\ 
     \bottomrule
\end{longtable}


\newpage 

\subsection*{Effect of harmonisation and trait aggregation on functional diversity metrics}

\begin{figure}[ht]
    \centering
    \includegraphics[width=17cm, height=12cm]{Correlation_fd_metrics_FRic.png}
    \caption{Fitted regressions between residuals of FRic with traits used in the original study and with harmonised and aggregated traits. Dots refer to each site-year combination. Sites have been sampled maximum three times in three years. For further details please refer to sections "Effects of harmonisation and trait aggregation on inferences regarding trait-environment relationships" and the original study.}
    \label{fig:FRic}
\end{figure}

\begin{figure}[ht]
    \centering
    \includegraphics[width=17cm, height=12cm]{Correlation_fd_metrics_FDiv.png}
    \caption{Fitted regressions between residuals of FDiv with traits used in the original study and with harmonised and aggregated traits. Dots refer to each site-year combination site-year combination. Sites have been sampled maximum three times in three years. Some sites were sampled multiple times within a year. In that case the highest sampling event with the highest total invertebrate abundance was taken (this explains the discrepancy to the original study, that did not consider multiple samplings). For further details please refer to sections "Effects of harmonisation and trait aggregation on inferences regarding trait-environment relationships" and the original study.}
    \label{fig:FDiv}
\end{figure}

\begin{figure}[ht]
    \centering
    \includegraphics[width=17cm, height=12cm]{Correlation_fd_metrics_FEve.png}
    \caption{Fitted regressions between residuals of FEve with traits used in the original study and with harmonised and aggregated traits. Dots refer to each site-year combination site-year combination. Sites have been sampled maximum three times in three years. Some sites were sampled multiple times within a year. In that case the highest sampling event with the highest total invertebrate abundance was taken (this explains the discrepancy to the original study, that did not consider multiple samplings). For further details please refer to sections "Effects of harmonisation and trait aggregation on inferences regarding trait-environment relationships" and the original study.}
    \label{fig:FEve}
\end{figure}


\end{document}