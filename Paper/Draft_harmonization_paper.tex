\documentclass{article}
\usepackage[utf8]{inputenc}
\usepackage{hyperref}
\usepackage{float}
\usepackage[table,xcdraw]{xcolor}
\usepackage{color, colortbl}
\usepackage{longtable}
%\usepackage[sort&compress,square,comma,authoryear]{natbib}
\usepackage{booktabs}
\usepackage{graphicx}
\usepackage{multirow}
% \graphicspath{{/home/kunz/Dokumente/Projects/Trait_DB/Invertebrate_traits/Paper/Figures/}}
\graphicspath{{Figures/}}
\usepackage{rotating}
\usepackage{geometry}
\usepackage{array}
\usepackage{lscape}
\usepackage{longtable}
\usepackage{hhline}
\usepackage[
backend = biber,
style = numeric,
citestyle = numeric-comp,
sorting = none % sort by name year title
]{biblatex}
\addbibresource{Ref_invertebrate_DB.bib}

% TODO: Change citation style

\usepackage{subfiles} % Best loaded last in the preamble

% functions and definitions
\definecolor{Gray}{gray}{0.9}

% horizontal space between two columns
\setlength{\tabcolsep}{2mm}

\newcommand{\specialcell}[2][c]{%
  \begin{tabular}[#1]{@{}c@{}}#2\end{tabular}}

\renewcommand*{\thefootnote}{\alph{footnote}}


%%%%%%%%%%%%%%%%%%%%%%%%%%%%%%%%%%%%%%%%%%%%%%%%%%%%%%%%%%%%%%%%%%%%%%%%%%%%%%%%%%
%%%%%%%%%%%%%%%%%%%%%%%%%%%%%%%%%%%%%%%%%%%%%%%%%%%%%%%%%%%%%%%%%%%%%%%%%%%%%%%%%%
\title{OVERVIEW RESULTS: Harmonized invertebrate grouping feature databases}
\author{}%Stefan Kunz 
\date{}

\begin{document}
\maketitle

\section*{Introduction}

Explaining and predicting how aquatic communities are shaped by environmental factors is a main goal of freshwater ecology. Organismal traits, defined as measurable properties of an organism (\cite{mcgill_rebuilding_2006}), are increasingly incorporated into freshwater ecology to support this goal, e.g. by relating macroinvertebrate trait composition to environmental factors or as trait metrics in biomonitoring (\cite{poff_developing_2010, szocs_effects_2014, bhowmik_large_2015, menezes_beyond_2010}). Traits evolve through adaptations (e.g., physiological, behavioral) of organisms to their environment and indicate direct or indirect linkages between biological response of an organism to its environment (\cite{southwood_habitat_1977, verberk_delivering_2013}). % TODO cite also Statzner and Beche 2010?
Besides providing a mechanistic explanation of species-environment relationships, trait-based approaches may be suitable for large scale analysis since variability in trait responses is less than for taxonomic responses (\cite{bonada_taxonomic_2007}).

In the last decades, freshwater ecologists developed comprehensive invertebrate trait databases for various biogeographic regions (\cite{usseglio-polatera_biomonitoring_2000, schmidt-kloiber_www.freshwaterecology.info_2015, vieira_database_nodate, Philips_and_Smith_NZ_DB_2018, kefford_integrated_2020, tomanova_trophic_2006}).
% ? Cite also Twarodochleb/
% Tomanova -> rather small database
%(see Table \ref{tab:overview_trait_dbs} for a summary). %? Reasoning/Motivation
The availability of invertebrate trait data from different biogeographic regions enables comparisons of trait variation and their relation to environmental factors across large scales. However, such analysis have been carried out mostly within one biogeographic region, using information from one or two trait databases. For example, \cite{bonada_taxonomic_2007} compared trait composition for mediterranean and temperate regions in Europe using traits from \cite{usseglio-polatera_biomonitoring_2000}, \cite{poff_developing_2010} characterized trait composition across sites in the Western US using traits from \cite{poff_functional_2006}, and \cite{botwe_effects_2018} investigated the effect of salinity on invertebrate traits in different sites in South Australia using trait data from \cite{poff_functional_2006} and \cite{schafer_trait_2011}.
% ? Include Saito for South America 
Rarely have analysis on invertebrate traits been carried out that synthesize information on traits and grouping features from more than two different biogeographic regions. A grouping feature is a general property (e.g. feeding mode) that comprises a "group of related traits (e.g., predator, shredder, etc.) that vary among species or among individuals within a species" (\cite{schmera_proposed_2015}). To our knowledge, only \cite{brown_functional_2018} harmonized grouping features from more than two geographically distant invertebrate trait databases in a study on the influence of decreasing glacier cover in nine biogeographic regions on functional diversity and community assembly of invertebrates. 
% TODO: short Literature search for similar studies like Brown

We suspect that the heterogeneity of descriptions of the invertebrate trait databases is likely to be one of the reasons why information of these databases is rarely combined. In order to harmonize trait data from different regions, first commonly accepted and unambiguous trait definitions are required (\cite{schneider_towards_2019}). In the best case, grouping features would be classified into the same traits across databases or they could easily be harmonized using standardized terminology. However, a lack of standardized terminology of trait definitions and poor metadata quality in many trait databases is a common issue throughout the field of trait-based ecology (\cite{baird_toward_2011, schneider_towards_2019}). Secondly, consistent coding of traits facilitates the compatibility of trait data from different databases. Traits can be binary (i.e. trait exists or not), continuous, or fuzzy coded variables. Fuzzy codes represent the affinity of an organism to express a certain trait. They are used to account for plasticity in traits and are usually converted to percentages. Continuous values are typically used for grouping features hat can be measured, like body size. However, invertebrate trait databases are heterogeneous with regard to the coding they use for their traits (\cite{culp_incorporating_2011}, see also Table \ref{tab:trait_databases_coding_differentiation}). \cite{brown_functional_2018} harmonized grouping features based on trait databases from Europe, North America, and New Zealand since in these trait databases identical grouping features are classified differently into traits depending on the database. Also, because traits from North America were differently coded (binary) than those from Europe and New Zealand (fuzzy coded), the authors consulted experts to assign traits from North America or inferred them from the European database. 
% TODO: Tax res. not high enough in the used databases -> aggr. observations
Thus, it becomes apparent that using invertebrate trait data from several regions requires extensive data processing prior to the actual data analysis.

% TODO: Look into modelling papers 
Differing taxonomic resolutions between the observed taxa in a study and the used trait database is another challenge when working with trait data. When observations are on a lower taxonomic level than data available in the trait databases (e.g. observations on species-level, trait data on genus-level) trait data of the higher taxonomic level are often assigned (e.g. \cite{szocs_effects_2014}). % TODO: another reference
Conversely, if trait information is only available on lower taxonomic levels than the observed taxa, traits are aggregated to a higher taxonomic level (e.g. \cite{poff_functional_2006, szocs_effects_2014, piliere_a._f._h._importance_2016}). Thereby, trait aggregation is often done by using the median or the mode. Up to now, studies on how and to which extent different trait aggregation methods compare to assessments by trait experts are missing.
%? cite Poff
%? Taxonomic coverage
% Piliere: Family Monitoring data -> Trait data only on Genus level
% Brown: Monitoring data on species level -> Trait on genus or subfamily

% Example Poff 2006; TODO check in modelling papers
% (Similar taxonomical coverage for the investigated traits across the used databases → depends on the study) 

% TODO: Make sentences less difficult
In this study we (1) present four harmonized invertebrate grouping feature databases for seven grouping features based on information from trait databases of the regions Europe, North America, New Zealand and Australia. %? Outline challenges of data preparation process
Furthermore, we (2) compare to family-level aggregated trait values obtained by different trait aggregation methods to trait values assigned at family-level by experts. We (3) analyze how the usage of harmonized and/or aggregated trait data alters the results compared to non harmonized and non aggregated trait data by re-analyzing data on the effect of anthropogenic salinization on biological traits by \cite{szocs_effects_2014}. Finally, we (4) present an overview of discrepancies in trait definition between the used invertebrate trait databases and discuss challenges of trait data synthesis.
% The next sections are organised as follows: we first describe the used trait databases and outline the data preparation process. Then different trait aggregation 

%%%%%%%%%%%%%%%%%%%%%%%%%%%%%%%%%%%%%%%%%%%%%%%%%%%%%%%%%%%%%%%%%%%%%%%%%%%%%%%%%%%%%%%%%%%%%%%%%%%%%%%%%%%%%%%%%%%%%%%%%%%%%%%%%%%%%%%%%%%%%%%%%%%%%%%%%%%%%%%%

\section*{Methods}

\subsection*{Selection of traits and harmonization of trait databases}

% TODO:
%  Nominal and continuous traits converted to binary traits 
%  Normalize fuzzy codes to range [0 – 1]
%  Transform trait values in percentages
% ? Show data processing using a figure

We established four harmonized grouping feature databases using information available in invertebrate trait databases from Europe, North America, Australia, and New Zealand. Trait information for Europe was obtained by the Freshwaterecology database (\cite{schmidt-kloiber_www.freshwaterecology.info_2015}) and complemented by Tachet (\cite{usseglio-polatera_biomonitoring_2000}) when information was missing (e.g. for the grouping feature size). Trait information for North America was obtained from Laura Twardochleb (\textit{in prep.}) and complemented by Vieira et al. 2006 (\cite{vieira_database_nodate}). Philippe Usseglio-Polatera provided information on body form for European and North American taxa. For Australia and New Zealand, we used trait databases from Kefford et al. (\cite{kefford_integrated_2020}) and Philips and Smith respectively (\cite{Philips_and_Smith_NZ_DB_2018}).

We selected traits of seven grouping features based on the availability of trait information across databases and so that they describe different parts of the biology of an organism: life history (Voltinism), morphology (Respiration, Body form, Size), ecology (Locomotion, Feeding mode) and reproduction (Oviposition). %? frequently used traits
We did not include ecological traits that describe habitat preferences (e.g. temperature preference) as these traits are not reported in the New Zealand trait database. Since the grouping features were differently classified across the databases we harmonized them into 26 traits (Table \ref{tab:traits_harmonization}). Harmonization was undertaken by amalgamating similar traits into one trait (e.g. crawlers and sprawlers into crawlers). Thereby, the highest trait affinity was assigned to the harmonized trait per taxa. 
% ?Problem of different definitions for the same traits between the used databases -> mention comparison 

We used fuzzy coded traits for establishing our harmonized databases unless data quality prohibited and then we used binary traits, i.e. categorical and continuous traits were converted into binary traits. Fuzzy codes are reported with different ranges in the trait databases (e.g. freshwaterecology 0 to 10, Tachet 0 to 3 or 0 to 5). We standardized them to a range between 0 and 1 and converted trait affinities to percentages. 
% ? fuzzy coded and binary traits are in the same range now

Prior harmonization we amalgamated duplicate taxa on species, genus or family-level if present. We omitted taxa with a lower taxonomic resolution than family-level.

% TODO: Present single invertebrate trait databases:
%  Which taxonomical groups are covered
%  Which traits are covered?
%  - (often correlated with environmental variables)
%  - (Relatively good taxonomical overlap for these traits between the used databases)
% Compare original databases in a table for better overview
% TODO: Mention scripts for data processing and analysis (two github repos)

\begin{table}[H]
\centering
\caption{Traits of the harmonized grouping features. The last column indicates traits that were amalgamated (no amalgamation needed if empty).}
\label{tab:traits_harmonization}
\begin{tabular}{lll}
\\
Grouping feature & Trait & Amalgamated traits\\
\toprule[.1em]
Voltinism    & \begin{tabular}[c]{@{}l@{}}Semivoltine\\ Univoltine\\ Bi/multivoltine\end{tabular}                                & \begin{tabular}[c]{@{}l@{}}\textless 1 generation per year\\ 1 generation per year\\ \textgreater 1 generation per year\end{tabular}                                                                                                                                            \\
\midrule
Body Form    & \begin{tabular}[c]{@{}l@{}}Cylindrical \\ Flattenend\\ Spherical\\ Streamlined\end{tabular}                       & \begin{tabular}[c]{@{}l@{}}Round (humped)\\ Dorsoventrally flattened\\ Tubular\\ Fusiform\end{tabular}                                                                                                                                                                          \\
\midrule
Size         & \begin{tabular}[c]{@{}l@{}}Small \\ Medium \\ Large\end{tabular}                                                  & \begin{tabular}[c]{@{}l@{}}\textless 9 mm (Tachet: \textless 10 mm)\\ 9 - 16 mm (Tachet: 10 - 20 mm)\\ \textgreater 16 mm (Tachet: \textgreater 20 mm)\end{tabular}                                                                                                                \\
\midrule
Respiration  & \begin{tabular}[c]{@{}l@{}}Gills\\ Plastron/Spiracle\\ \\ \\ Tegument\end{tabular}                             & \begin{tabular}[c]{@{}l@{}}Temporary air store, tracheal gills\\ Spiracular gills, atmospheric breathers, \\ plant breathers, functional spiracles, \\ air (plants), aerial\\ Cutaneous\end{tabular}                                                                         \\
\midrule
Locomotion   & \begin{tabular}[c]{@{}l@{}}Burrower\\ Crawler\\ Sessil\\ Swimmer \end{tabular}                                     & \begin{tabular}[c]{@{}l@{}}Interstitial, boring\\ Sprawler, walking, climber, clinger\\ Attached\\ Skating, diving, planctonic\end{tabular}                                                                                                                 \\
\midrule
Feeding mode & \begin{tabular}[c]{@{}l@{}}Filterer\\ \\ Gatherer\\ \\ Herbivore\\ \\ Parasite\\ Predator\\ Shredder\end{tabular} & \begin{tabular}[c]{@{}l@{}}Active/passive filterer, absorber, \\ filter-feeder, collector-filterer\\ Deposit-feeder, collector-gatherer, \\ detrivore\\ Grazer, scraper, piercer herbivore, \\ algal piercer\\ \\ Piercer \\ Miner, xylophagus, shredder detrivore\end{tabular} \\
\hline
Oviposition  & \begin{tabular}[c]{@{}l@{}}Aquatic eggs\\ \\ Ovoviviparity\\ Terrestrial eggs\end{tabular}                            & \begin{tabular}[c]{@{}l@{}}Eggs attached to substrate/plants/stones,\\ free/fixed eggs/clutches\\ \\ Terrestrial clutches\end{tabular}     \\
\bottomrule[.1em]
\end{tabular}
\end{table}

%%%%%%%%%%%%%%%%%%%%%%%%%%%%%%%%%%%%%%%%%%%%%%%%%%%%%%%%%%%%%%%%%%%%%%%%%%%%%%%%%%%%%%%%%%%%%%%%%%%%%%%%%%%%%%%%%%%%%%%%%%%%%%%%%%%%%%%%%%%%%%%%%%%%%%%%%%%%%%%%
\subsection*{Trait aggregation}

Traits of the harmonized grouping feature databases were aggregated to family-level using three approaches. I) we directly aggregated taxa to family-level giving equal weight to every species. We denote this aggregation as \textit{direct\_agg}. For the \textit{direct\_agg} we tested aggregating with the median and the mean. We added \textit{median} or \textit{mean} to \textit{direct\_agg} to indicate when we used which method. II) taxa were aggregated stepwise, i.e first to the genus-level and subsequently to the family-level. By using this approach, we give equal weights to each genus. Hereafter, we abbreviate this aggregation type as \textit{stepwise\_agg}. We tested the \textit{stepwise\_agg} using the mean and the median, using the same naming as for the \textit{direct\_agg}. III) taxa were aggregated using a weighted mean approach, denoted as \textit{weighted\_agg}. This method weights the genera according to the number of their species present in the databases. As mentioned above, trait affinities ranged between 0 and 1. Hence, the maximum differences possible is 1 or -1 (corresponds to 100 \%). For convenience, we report absolute trait differences.

%%%%%%%%%%%%%%%%%%%%%%%%%%%%%%%%%%%%%%%%%%%%%%%%%%%%%%%%%%%%%%%%%%%%%%%%%%%%%%%%%%%%%%%%%%%%%%%%%%%%%%%%%%%%%%%%%%%%%%%%%%%%%%%%%%%%%%%%%%%%%%%%%%%%%%%%%%%%%%%%

\subsection*{Data analysis with harmonized grouping features and aggregated traits}

To investigate how harmonizing grouping features and aggregating invertebrate traits might change the results in the analysis of trait-environment relationships we replicated the data analysis of Szöcs et al. 2014 (\cite{szocs_effects_2014}) using the harmonized grouping features \textit{Body size, Feeding mode, Locomotion, Reproduction/Oviposition, Respiration} and \textit{Voltinism} (21 grouping features have been used in total) and additionally aggregated traits using the aforementioned aggregation methods. The harmonized grouping features used are those that responded strongly to salinity in the study of Szöcs et. al. 2014, except for life cycle duration. For testing the effect of aggregated traits we assigned to each taxon in \cite{szocs_effects_2014} the aggregated trait value from the established harmonized European grouping feature database for its corresponding family. We limit our analysis to the RDA of traits constrained by electric conductivity of the original publication (see appendix \ref{subsec:SI_szoecs_reanalysis} for a more in-depth comparison to the original results). An overview of the harmonization for the European trait databases can be found in the supporting information in section \ref{sec:SI_harmonization_EU}.

%   We want replicate the trait analysis in an example study using our harmonized traits. Therefore, we use the study from Szöcs et al. 2014 on the effect of anthropogenic salinization on biological traits. For this small scale study we re-analyze the change in trait composition along the conductivity gradient using the traits we could harmonize. 

%\section{Results \& Discussion}

%%%%%%%%%%%%%%%%%%%%%%%%%%%%%%%%%%%%%%%%%%%%%%%%%%%%%%%%%%%%%%%%%%%%%%%%%%%%%%%%%%%%%%%%%%%%%%%%%%%%%%%%%%%%%%%%%%%%%%%%%%%%%%%%%%%%%%%%%%%%%%%%%%%%%%%%%%%%%%%%

\newpage
\section*{Results harmonized grouping feature databases}

We used information on invertebrate traits for the regions Europe, North America, Australia, and New Zealand to establish harmonized grouping feature databases. Trait information for Europe was obtained by the Freshwaterecology database (\cite{schmidt-kloiber_www.freshwaterecology.info_2015}) and complemented by Tachet (\cite{usseglio-polatera_biomonitoring_2000}) when information was missing (e.g. for the grouping feature size). Trait information for North America was obtained from Laura Twardochleb and complemented by Vieira et al. 2006 (\cite{vieira_database_nodate}). Philippe Usseglio-Polatera provided information on body form for European and North American taxa. For Australia and New Zealand, we used trait databases from Kefford et al. (\cite{kefford_integrated_2020}) and Philips and Smith respectively (\cite{Philips_and_Smith_NZ_DB_2018}).

We selected seven grouping features that we harmonized into 26 traits. The grouping features were chosen based on the availability of trait information across databases and so that they describe different parts of the biology of an organism: life history (Voltinism), morphology (Respiration, Body form, Size), ecology (Locomotion, Feeding mode) and reproduction (Oviposition).
We used fuzzy coded traits for establishing our harmonized databases unless data quality prohibited and then we used binary traits, i.e categorical and continuous traits were converted into binary traits. Data processing prior harmonization also involved amalgamation of duplicate taxa on species, genus or family-level and conversion of trait affinities to percentages per trait. We omitted taxa with a lower taxonomic resolution than family-level. The following results relate to the harmonized grouping feature databases.

%! Completeness of trait information and taxonomical coverage
\subfile{Results/Taxonomic_cov_and_info_compl}

\newpage 
%! Comparison of trait aggregation methods
\subfile{Results/Results_trait_agg}

%! Presenting results of trait analysis using harmonized traits:
%  Did using harmonized traits yield different results regarding trait composition of down and upstream sites in Szöcs et al. 2014?
%  Had various trait aggregation methods an impact?
\newpage 

\subfile{Results/Results_trait_harmonization}

\newpage 

%! Discrepancies in trait definitions
% ?(Discuss history of trait values)
\subfile{Trait_def_table/Trait_definitions_table}

\section*{Discussion}

% Stat. techniques developed -> cannot use full potential
% if data quality is poor, when synthesizing information is hard 

% Bibliography
\printbibliography
\newpage

%! SI
\subfile{Results/SI}

% Trait definitions 
\subfile{Trait_def_table/SI_trait_def_table2.tex}

\end{document}

%%%%%%%%%%%%%%%%%%%%%%%% Examples of trait databases %%%%%%%%%%%%%%%%%%%%%%%%%%%%
% % ? Also mention Tomanova? -> not used here
% \begin{table}[ht]
%   \centering
%     \caption{Examples of aquatic invertebrate trait databases established in the last decades. Differences in codings and trait definitions are outlined for these databases in Table \ref{tab:trait_databases_coding_differentiation} and Table S\ref{tab:trait_definitions}.}
%     \label{tab:overview_trait_dbs}
%     \begin{tabular}{lcccl}
%     \hline
%     Database & \specialcell{Nr. of grouping \\ features} & \specialcell{Main organism \\ groups} & \specialcell{Taxa and \\ resolution} & Citation \\ 
%     \hline
%     Freshwaterecology & 40 & \specialcell{Invertebrates, biased \\ towards aquatic insects} & Species-level & \cite{schmidt-kloiber_www.freshwaterecology.info_2015} \\ 
%     Tachet & 21 & \specialcell{Invertebrates, biased \\ towards aquatic insects} & \specialcell{Taxa on species, \\ genus, and family-level} & \cite{usseglio-polatera_biomonitoring_2000} \\ 
%     \specialcell{North America \\ (Vieira)} & 62 & \specialcell{Invertebrates, biased \\ towards aquatic insects} & \specialcell{Taxa on species, \\ genus, and family-level} & \cite{vieira_database_nodate} \\ 
%     \specialcell{North America \\ (Twardochleb)} & 11 & Aquatic insects & \specialcell{Taxa on species, \\ genus, and family-level} & \cite{vieira_database_nodate} & \textit{In prep.} \\ 
%     Australia & & \specialcell{Invertebrates, biased \\ towards aquatic insects} & \specialcell{Taxa on species, \\ genus, and family-level} & \cite{kefford_integrated_2020} \\
%     New Zealand & & & & \cite{Philips_and_Smith_NZ_DB_2018}\\
%     \hline
%     \end{tabular}
% \end{table}
%%%%%%%%%%%%%%%%%%%%%%%%%%%%%%%%%%%%%%%%%%%%%%%%%%%%%%%%%%%%%%%%%%%%%%%%%%%%%%%%%