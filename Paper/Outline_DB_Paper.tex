\documentclass{article}
\usepackage[utf8]{inputenc}
\usepackage{hyperref}
\usepackage{float}
\usepackage[table,xcdraw]{xcolor}
%\usepackage[sort&compress,square,comma,authoryear]{natbib}
\usepackage{booktabs}
\usepackage{graphicx}
\graphicspath{ {"/home/kunz/Dokumente/Projects/Trait_DB/Invertebrate_traits/Paper/Figures"} }
\usepackage{longtable}


\usepackage[
backend=biber,
style=authoryear,
sorting=nyt % sort by name year title
]{biblatex}
\addbibresource{Ref_invertebrate_DB.bib}

\title{ OUTLINE: Harmonized macroinvertebrate trait database, Aggregation of traits, Trait definitions, Sources }
\author{}%Stefan Kunz 
\date{}%June 2019


\begin{document}
%TODO: Use citep & citet instead of cite
\maketitle

\section{Introduction}

Intro: For what are traits used?
\\
\\
\\
Knowledge on macroinvertebrate traits $\rightarrow$ trait databases
\\
\\
\\
Goal: 
We harmonized information from different trait databases of four regions in the world,
namely: Australia, Europe, North America, and New Zealand. \\
\\
Studies that use information on aquatic invertebrate traits from different regions and/or aggregate trait information 
are increasing %! Change wording & State examples: Brown Paper
\\
\\
In this paper we examine difficulties that ecologists face when using different macroinvertebrate trait databases together. 
We explore the effect of different decisions researches have to make when working with trait data from several regions/sources, 
involving harmonization, handling different codings, normalization, and aggregation of traits. 
Therefore, we harmonized six traits from four trait databases and aggregated the trait information to family level. % ?Name Traits
Our paper also compares the references for the used trait information that were specified in the used trait databases. 
Finally, we present an overview of differences in trait definitions among databases. 

That means: 
\begin{itemize}
    \item How to harmonize different trait databases when the task involves using 
    traits from different regions (e.g. for comparing trait profiles across regions) $\rightarrow$ Harmonization
    \item Thereby, we will examine differences in trait definitions
    \item Look on the sources of various trait databases 
    \item Furthermore, show effect of different ways of aggregating traits (inter alia Problem of different
     coding styles (fuzzy vs binary))
\end{itemize}

\section{Description of harmonized trait database} %TODO: Change heading to more expressive

% Mention restriction to certain orders
The harmonized database consists of the available information on aquatic invertebrate traits for the regions 
Europe, North America, Australia, and New Zealand and comprises the traits locomotion, feeding mode, respiration, voltinism, 
size, and body form. The pattern of development (holometabolous or hemimetabolous) was added as an additional trait 
based on the orders of the taxa included in each database. 
Traits for Europe were retrieved from the freshwaterecology trait database (\url{https://www.freshwaterecology.info/}) 
and complemented by information from Tachet (\cite{usseglio-polatera_biomonitoring_2000}). 
North American macroinvertebrate traits were taken from Laura Twardochleb and complemented by trait information from Viera (\cite{vieira_database_nodate}). 
% TODO: Insert citation
An overview of the used databases can be found in table \ref{tab:trait_databases}. \\ 
In the following paragraphs, we describe the data processing steps required to establish a harmonized 
macroinvertebrate trait database. 

\begin{table}[H]
    \centering
    \caption{Overview of trait databases.}
    \label{tab:trait_databases}
    \begin{tabular}{lll}
    \toprule
   Region & Coding of trait states & Reference \\ 
    \hline
   Europe & Largely fuzzy & \cite{schmidt-kloiber_www.freshwaterecology.info_2015}\\ 
   Central Europe & Fuzzy coded & \cite{usseglio-polatera_biomonitoring_2000} \\ 
   North America & Largely binary & \cite{vieira_database_nodate}\\
   North America & Largely binary & cite Laura Twardochleb \\
   Australia & Binary \& fuzzy coding  & \cite{kefford_ben_AST_DB_2019}\\ 
   New Zealand & Fuzzy coded & \\ %TODO: Create entry for more recent NZ reference
    \bottomrule
    \end{tabular}
\end{table}


\section{Harmonization of traits}

\begin{itemize}
    \item Harmonization process: See also Schmera 2015 et al 
    \item Differences in trait definitions
    \item Sources of traits
\end{itemize}

\section{Aggregation of traits}

\begin{itemize}
    \item Describing \& testing different approaches
    \item Problem of coding of traits
\end{itemize}

\section*{Additional ideas}

Section Description of databases:
\begin{itemize}
    \item Describe different databases briefly?
    \item State goal of analysis $\rightarrow$ reference to second paper?
\end{itemize}

\end{document}